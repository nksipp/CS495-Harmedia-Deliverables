% SRD_V1.tex - Software Requirements Document for CS495
% Last Updated: 2-15-16

%preamble section (?)
%-----------------------------%

% Adding own custom commands
%none yet!

%indent first package
%\usepackage{indentfirst}

\documentclass[12pt,a4paper,onesie,titlepage,draft]{article}
%-----------------------------%

\begin{document}
\title{Harmedia's Software Requirements Document}
\author{Christian Brewton, Austin Kelley, Norman Sipp}
\date{Last Updated: \today}
\maketitle

\begin{enumerate}
	\item Introduction
		\begin{enumerate}
			\item Purpose \\
				This document describes the software requirements for a web based video synchronization platform called Harmedia.  This specifications is intended for the designer, developer and maintainer of the Harmedia system.
			\item Scope \\
				The Harmedia video synchronization platform \textbf{HVSP} will be a web based platform that is used to synchronize video for multiple clients.  Users will be able to create their own channel and moderate/manage it.  Within each channel there will be a video queue where elected people can remove, insert, and modify videos in the queue.  Furthermore, in every channel there can be a channel guru and they will have access to modified the video being synchronized state.
			\item Overview of Document \\
				The remainder of this document will be organized as followed: In the following section there will be definitions that are vital for one to understand how HVSP will work.  Chapter 2 will contain the general descriptions of HVSP.  Lastly, Chapter 3  will identify the specific functional requirements, the external interfaces and the performance requirements of the HVSP.
			\item Definitions
				\begin {itemize}
					\item \textbf{Definition 1} \\
						\textit{Definition goes here!}
				\end {itemize}
		\end{enumerate}
	\item General Descriptions
		To give a short overview of the functionality of the HVSP the following user scenarios are provided:
		\begin{itemize}
			\item User System \\
				\underline{\textit{Sign Up:}}
			\begin{enumerate}
				\item Stuff
			\end{enumerate}
			\underline{\textit{Log-in:}}
			\begin{enumerate}
				\item Stuff
			\end{enumerate}
			\item Channel/Room Browser\\
				\underline{\textit{Search Room By Name:}}
				\begin{enumerate}
					\item Stuff
				\end{enumerate}
			\item Channel/Room \\
				\underline{\textit{Chat:}}
				\begin{enumerate}
					\item Stuff
				\end{enumerate}
		\end {itemize}	
	\item Requirements
\end{enumerate}
\end{document}

